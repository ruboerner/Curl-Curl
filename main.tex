
\documentclass[fleqn,aspectratio=169]{beamer}
\let\Tiny=\tiny
\usepackage[ngerman]{babel}
\usepackage[]{amsmath,amssymb,esdiff,esint}
\usepackage{physics}
\usepackage{graphicx}
\usepackage[scale=2]{ccicons}

\usetheme[%%
progressbar=frametitle,%%
numbering=none]{metropolis}

\begin{document}

\begin{frame}{Herleitung Rot-Rot-Gleichung}
    Ziel:
    
    \begin{itemize}
        \item Elimination einer der beiden Feldgrößen in Maxwellgleichung
        \item Aufstellen einer partiellen Differentialgleichung zweiter Ordnung
    \end{itemize}
    
\end{frame}

\begin{frame}{Herleitung Rot-Rot-Gleichung}
    \begin{columns}[c]
    \begin{column}{0.5\textwidth}
    \begin{equation*}
        \only<1>{\curl \vb{e} + \pdv{\vb{b}}{t} = \vb{0}}
        \only<2>{\curl \curl \vb{e} + \curl \pdv{\vb{b}}{t} = \vb{0}}
        \only<3>{\curl \curl \vb{e} + \curl \pdv{\textcolor{blue}{\vb{b}}}{t} = \vb{0}}
        \only<4>{\curl \curl \vb{e} +  \pdv{t}\curl \mu \vb{h} = \vb{0}}
        \only<5>{\curl \curl \vb{e} +  \pdv{t} \mu \curl \vb{h} = \vb{0}}
        \only<6>{\curl \curl \vb{e} +  \mu \pdv{t} \curl \vb{h} = \vb{0}}
        \only<7>{\curl \curl \vb{e} +  \mu \pdv{t} \textcolor{blue}{\curl \vb{h}} = \vb{0}}
        \only<8>{\curl \curl \vb{e} +  \mu \pdv{t}
        \left(\textcolor{blue}{\vb{j} + \pdv{\vb{d}}{t}}\right) = \vb{0}}
        \only<9>{\curl \curl \vb{e} +  \mu \pdv{t}
        \left(\sigma \vb{e} + \pdv{t} \varepsilon \vb{e}\right) = \vb{0}}
        \only<10>{\curl \curl \vb{e} +  \mu \pdv{t}
        \left(\sigma \vb{e} + \varepsilon \pdv{t}  \vb{e}\right) = \vb{0}}
        \only<11>{\curl \curl \vb{e} +  \mu\sigma \pdv{\vb{e}}{t}
        + \mu \varepsilon \pdv[2]{\vb{e}}{t}   = \vb{0}}
    \end{equation*}
    \end{column}
    \quad
    \begin{column}{0.44\textwidth}
    \only<1>{Faradaysches Induktionsgesetz}
    \only<2>{Rotation}
    \only<3>{Materialgleichung $\vb{b} = \mu \vb{h}$}
    \only<4>{\par Vertauschen der Reihenfolge von Rotation und zeitlicher Ableitung}
    \only<5>{$\mu$ homogen}
    \only<6>{$\mu$ zeitunabhängig}
    \only<7>{\par Elimination von $\vb{h}$}
    \only<8>{\par Amperesches Durchflutungsgesetz $\curl \vb{h} = \vb{j} + \pdv{\vb{d}}{t}$}
    \only<9>{\par Ohmsches Gesetz $\vb{j} = \sigma \vb{e}$ \\ Materialgleichung $\vb{d} = \varepsilon \vb{e}$}
    \only<10>{\par $\sigma$ und $\varepsilon$ zeitunabhängig}
    \only<11>{\par Partielle Differentialgleichung zweiter Ordnung für $\vb{e}$}
    \end{column}
    \end{columns}
\end{frame}

\end{document}
