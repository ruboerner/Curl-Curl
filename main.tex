%!TEX TS-program = lualatex
\documentclass[fleqn,aspectratio=43]{beamer}
\let\Tiny=\tiny
\usepackage[ngerman]{babel}
\usepackage[]{amsmath,amssymb,esdiff,esint}
\usepackage{physics}
\usepackage{graphicx}
\usepackage{lmodern}
\usepackage[scale=2]{ccicons}

\definecolor{myblue}{rgb}{0.0,0.3921,0.6588}

\usetheme[%%
progressbar=none,%%
numbering=none]{metropolis}
\setsansfont[BoldFont={Fira Sans}]{Fira Sans Light}
\setmonofont{Fira Mono}
\setbeamercolor{frametitle}{bg=myblue}
\setbeamercovered{transparent}
\begin{document}

\begin{frame}{Herleitung Rot-Rot-Gleichung}
    Ziel:
    \begin{itemize}
        \item Elimination einer der beiden Feldgrößen in Maxwellgleichung
        \item Aufstellen einer partiellen Differentialgleichung zweiter Ordnung
    \end{itemize}

    Hilfsmittel:
    \begin{itemize}
        \item Rotation
        \item Homogenität und Stationarität der Materialparameter
    \end{itemize}

\end{frame}

\begin{frame}{Herleitung Rot-Rot-Gleichung}
    \begin{columns}[c]
        \begin{column}{0.5\textwidth}
            \begin{equation*}
                \only<1>{\curl \vb{e} + \pdv{\vb{b}}{t} = \vb{0}}
                \only<2>{\curl \curl \vb{e} + \curl \pdv{\vb{b}}{t} = \vb{0}}
                \only<3>{\curl \curl \vb{e} + \curl \pdv{\textcolor{blue}{\vb{b}}}{t} = \vb{0}}
                \only<4>{\curl \curl \vb{e} +  \pdv{t}\curl \mu \vb{h} = \vb{0}}
                \only<5>{\curl \curl \vb{e} +  \pdv{t} \mu \curl \vb{h} = \vb{0}}
                \only<6>{\curl \curl \vb{e} +  \mu \pdv{t} \curl \vb{h} = \vb{0}}
                \only<7>{\curl \curl \vb{e} +  \mu \pdv{t} \textcolor{blue}{\curl \vb{h}} = \vb{0}}
                \only<8>{\curl \curl \vb{e} +  \mu \pdv{t}
                    \left(\textcolor{blue}{\vb{j} + \pdv{\vb{d}}{t}}\right) = \vb{0}}
                \only<9>{\curl \curl \vb{e} +  \mu \pdv{t}
                    \left(\sigma \vb{e} + \pdv{t} \varepsilon \vb{e}\right) = \vb{0}}
                \only<10>{\curl \curl \vb{e} +  \mu \pdv{t}
                    \left(\sigma \vb{e} + \varepsilon \pdv{t}  \vb{e}\right) = \vb{0}}
                \only<11>{\curl \curl \vb{e} +  \mu\sigma \pdv{\vb{e}}{t}
                    + \mu \varepsilon \pdv[2]{\vb{e}}{t}   = \vb{0}}
            \end{equation*}
        \end{column}

        \quad

        \begin{column}{0.3\textwidth}
            \only<1>{\par Faradaysches Induktionsgesetz}
            \only<2>{\par Rotation}
            \only<3>{\par Materialgleichung $\vb{b} = \mu \vb{h}$}
            \only<4>{\par Vertauschen der Reihenfolge von Rotation und zeitlicher Ableitung}
            \only<5>{\par $\mu$ homogen}
            \only<6>{\par $\mu$ zeitunabhängig}
            \only<7>{\par Elimination von $\vb{h}$}
            \only<8>{\par Amperesches \\ Durchflutungs\-gesetz \\ $\curl \vb{h} = \vb{j} + \pdv{\vb{d}}{t}$}
            \only<9>{\par Ohmsches Gesetz \\ $\vb{j} = \sigma \vb{e}$ \\ Materialgleichung \\ $\vb{d} = \varepsilon \vb{e}$}
            \only<10>{\par $\sigma$ und $\varepsilon$ zeitunabhängig}
            \only<11>{\par Partielle \\ Differential\-gleichung zweiter Ordnung für $\vb{e}$}
        \end{column}
    \end{columns}
\end{frame}

\begin{frame}{Zerfall freier Ladungen}
    Ziel:
    \begin{itemize}
        \item Begründung für Quellfreiheit des elektrischen Feldes in homogen leitfähigen Medien
        \item Herleitung einer gewöhnlichen Differentialgleichung für die elektrische Ladungsträgerdichte $\rho_E(t)$
        \item Abschätzung der Zeitabhängigkeit der Ladungsträger\-dichte
    \end{itemize}
    Hilfsmittel:
    \begin{itemize}
        \item Divergenz
        \item Vektoridentitäten für Ableitungen
    \end{itemize}

\end{frame}

\begin{frame}{Zerfall freier Ladungen}
    \begin{columns}[c]
        \begin{column}{0.5\textwidth}
            \only<1>{
                \begin{equation*}
                    \curl \vb{h} - \pdv{\vb{d}}{t} = \vb{j}
                \end{equation*}
            }
            \only<2>{
                \begin{equation*}
                    \div \curl \vb{h} - \div \pdv{\vb{d}}{t} = \div \vb{j}
                \end{equation*}
            }
            \only<3>{
                \begin{equation*}
                    - \pdv{t} \div \vb{d} = \div \vb{j}
                \end{equation*}
            }
            \only<4>{
                \begin{equation*}
                    - \pdv{\rho_E}{t}  = \div \left(\sigma \vb{e}\right)
                \end{equation*}
            }
            \only<5>{
                \begin{equation*}
                    - \pdv{\rho_E}{t}  = \sigma \div  \vb{e}
                \end{equation*}
            }
            \only<6>{
                \begin{equation*}
                    - \pdv{\rho_E}{t}  = \sigma \frac{\rho_E}{\varepsilon}
                \end{equation*}
            }
            \only<7>{
                \begin{equation*}
                    \pdv{t}\rho_E  +  \frac{\sigma}{\varepsilon} \rho_E = 0
                \end{equation*}
            }
            \only<8>{
                \begin{equation*}
                    \rho_E(t)  =
                    \rho_E(0)
                    e^{-\dfrac{\sigma}{\varepsilon} t}
                \end{equation*}
            }
        \end{column}
        \quad
        \begin{column}{0.3\textwidth}
            \only<1>{\par Amperesches \\ Durchflutungs\-gesetz}
            \only<2>{Divergenz}
            \only<3>{\par Vertauschen der Ableitungen, $\div \curl (\cdot) = 0$}
            \only<4>{\par Ohmsches Gesetz, \\
                Gaußsches Gesetz für elektrische Felder}
            \only<5>{\par Homogenes Medium}
            \only<6>{\par Gaußsches Gesetz für elektrische Felder}
            \only<7>{\par Gewöhnliche \\ Differential\-gleichung für $\rho_E(t)$}
            \only<8>{Ansatz}
        \end{column}
    \end{columns}
\end{frame}

\begin{frame}{Zerfall freier Ladungen}
    Frage: Wann gilt
    \begin{equation*}
        \dfrac{\rho_E(t)}{\rho_E(0)} \ll 1,
    \end{equation*}
    wenn $\varepsilon = \varepsilon_0$ und
    $\sigma > 10^{-4}$ S/m?
\end{frame}

\setbeamercolor{background canvas}{bg=myblue}
\begin{frame}
    % \frametitle{Wurzeln von $k^2$}
    \textcolor{white}{\LARGE Die Quadratwurzeln von $k^2$}

\end{frame}

\setbeamercolor{background canvas}{bg=white}

\begin{frame}{Wurzeln von $k^2$}
\uncover<1->{    Ausgangspunkt: Helmholtzgleichung $\nabla^2 \vb{E} + k^2 \vb{E} = 0$ mit der Lösung
    $$
        \vb{E}(z) = \vb{E}(0) e^{- i k z} , \quad z \ge 0.
    $$
    Problem: $k = \textcolor{red}{\pm} \sqrt{-i \omega \mu \sigma}$
}

\uncover<2->{    Ziel: Auswahl der Wurzel so, dass
    \begin{itemize}
        \item die Felder $\vb{E}$, $\vb{B}$ usw. für $z \to +\infty$ verschwinden
        \item die Ausbreitungsgeschwindigkeit die physikalisch korrekte Richtung (\glqq nach unten laufende Welle\grqq, \glqq in positive z-Richtung laufende Welle\grqq) besitzt.
    \end{itemize}
}
\end{frame}

\begin{frame}{Wurzeln von $k^2$}
    \uncover<1->{Bekannt: \textcolor<2>{red}{$k^2 = -i \omega \mu \sigma$} (quasistationäre Näherung)

    Wir machen Ansatz: $k^2 = (\alpha - i\beta)^2 = \textcolor<3>{blue}{\alpha^2} \textcolor<3>{red}{- 2 i \alpha \beta} \textcolor<3>{blue}{- \beta^2}$.}

    \uncover<2->{Da $k^2$ \textcolor<2>{red}{rein imaginär} ist, muss $\alpha^2 - \beta^2 = 0$ sein.}

    \uncover<3->{Wir zerlegen in \textcolor<3>{blue}{Real-} und \textcolor<3>{red}{Imaginärteil}:
    \begin{align*}
        Re: \quad & \alpha^2 - \beta^2  = 0               \\
        Im: \quad & -2 \alpha \beta  \,\,= -\omega \mu \sigma
    \end{align*}}
    
    \uncover<4->{Wir erhalten zunächst $\beta = \dfrac{\omega\mu\sigma}{2\alpha}$.}

\end{frame}

\begin{frame}{Wurzeln von $k^2$}
\uncover<1->{    Mit dem Zwischenergebnis $\beta = \frac{\omega\mu\sigma}{2\alpha}$ kehren wir zurück zum Realteil von $k^2$. Wir setzen ein:
}
    \begin{align*}
        \uncover<1->{\alpha^2 - \left(\frac{\omega\mu\sigma}{2\alpha}\right)^2 & = 0}                                                             \\
        \uncover<2->{\alpha^4 - \left(\frac{\omega\mu\sigma}{2}\right)^2       & = 0}                                                             \\
        \uncover<3->{\alpha^2                                                  & = \textcolor<3>{red}{\pm} \frac{\omega\mu\sigma}{2} = \beta^2}                \\
        \uncover<4->{\alpha                                                    & = \textcolor<5>{red}{\pm} \sqrt{\textcolor<4>{red}{+}\frac{\omega\mu\sigma}{2}} = \beta}
    \end{align*}
    \uncover<5->
    {Die Entscheidung für das \textcolor<5>{red}{richtige Vorzeichen} fällt, wenn wir $\alpha$ und $\beta$ in die Lösung der eindimensionalen Helmholtzgleichung einsetzen.}
\end{frame}

\begin{frame}{Wurzeln von $k^2$}
    \uncover<1->{Wir erinnern uns, dass $k = \pm (\alpha - i \beta)$.}

    \uncover<2->{Es ist klar, dass in
\begin{align*}
    \vb{E}(z) & = \vb{E}(0) e^{- i k z}, \qquad i k = \pm i \alpha \pm \beta \\
    \vb{E}(z) & = \vb{E}(0) e^{- i \alpha z} e^{- \beta z} 
\end{align*}
nur $\beta > 0$ das geforderte Verhalten $\vb{E}(z) \to 0$ für $z \to +\infty$ liefert.}

\uncover<3->{$\beta$ bestimmt die Stärke des exponentiellen Abklingverhaltens von $\vb{E}(z)$ für $z \ge 0$, also die Dämpfung.}

\uncover<4->{Welche Rolle spielt $\alpha$?}
\end{frame}

\begin{frame}{Wurzeln von $k^2$}
\uncover<1->{    Zum Verständnis der Rolle von $\alpha$ müssen wir die Zeitabhängigkeit explizit in die Gleichung für $\vb{E}(z)$ aufnehmen:
}
\begin{align*}
    \uncover<1->{\vb{E}(z) & = \vb{E}(0) e^{- i(\textcolor<2-3>{red}{\alpha z - \omega t}) } e^{- \beta z}} \\
    \uncover<2->{\vb{E}(z) & = \vb{E}(0) e^{- i \textcolor<2-3>{red}{\varphi} } e^{- \beta z}}
\end{align*}
\uncover<3->{\textcolor<3>{red}{$\varphi$} ist ein Phasenwinkel und beschreibt den Zustand der Amplitude als Funktion von Ort ($z$) und Zeit ($t$):
\begin{align*}
    \alpha z - \omega t & = \varphi \\
    z & = \frac{\varphi + \omega t}{\alpha}.
\end{align*}
}
\uncover<4->
{Daraus erhalten wir die Phasengeschwindigkeit
\[
    \dot z(t) = c = \frac{\omega}{\alpha}.
\]}
\end{frame}

\begin{frame}{Wurzeln von $k^2$}
\uncover<1->{Die Phasengeschwindigkeit
$$
c = \frac{\omega}{\textcolor<2>{red}{\alpha}} = \sqrt{\frac{2 \omega }{\mu \sigma}}
$$
besitzt nur für $\alpha > 0$ die Bedeutung der Geschwindigkeit einer sich in positiver $z$-Richtung ausbreitenden Welle.}

\uncover<2->{Wir erkennen in $\alpha$ die \textcolor<2>{red}{Wellenzahl}.}

\uncover<3->{Bemerkung: Für eine in negative $z$-Richtung laufende Welle kehren sowohl $\alpha$ als auch $\beta$ ihre Vorzeichen um!}
\end{frame}

\begin{frame}
    \frametitle{Wurzeln von $k^2$}

    Resultate:

    \uncover<1->{Wir können einerseits aus $\beta$ die elektromagnetische Eindringtiefe $\tau$ berechnen: $\tau = \beta^{-1} = \sqrt{\dfrac{2}{\omega \mu \sigma}}$.}

    \uncover<2->{Die Phasengeschwindigkeit können wir ebenfalls genau angeben: $c = \sqrt{\dfrac{2 \omega}{\mu \sigma}}$.}

    \uncover<3->{Andererseits zeigt uns die Phasengeschwindigkeit aber auch, dass wir mit der quasistationären Näherung für $\sigma \to 0$ ein wichtiges physikalisches Gesetz ($c$ stets kleiner als Vakuumlichtgeschwindigkeit $c_0$) verletzen!}
 
\end{frame}

\end{document}
